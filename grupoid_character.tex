\documentclass[11pt]{article}

\usepackage[T2A]{fontenc}
\usepackage[utf8]{inputenc}
\usepackage[russian]{babel}

\usepackage{indentfirst}
\usepackage{geometry}

\usepackage{amsmath, amssymb, amsfonts, amsthm, amscd}
\usepackage{mathrsfs}

\geometry{a4paper}
\sloppy


\theoremstyle{plain}
\newtheorem*{theorem}{Теорема}
\theoremstyle{definition}
\newtheorem*{definition}{Определение}
\theoremstyle{remark}
\newtheorem*{problem}{Задача}

\renewcommand{\le}{\leqslant}
\renewcommand{\ge}{\geqslant}
\renewcommand{\phi}{\varphi}
\renewcommand{\epsilon}{\varepsilon}
\renewcommand{\kappa}{\varkappa}

\newcommand{\comdots}{,\!..}
\newcommand{\subdim}[1]{\lfloor \dim #1 \rfloor}

\DeclareMathOperator*{\Hom}{Hom}
\DeclareMathOperator*{\Tor}{Tor}
\DeclareMathOperator*{\Obj}{Obj}
\DeclareMathOperator*{\Fund}{Fund}

\begin{document}

\author{А.~A.~Владимиров}
\title{Характер группоида}
\date{24.04.2022}
\maketitle

\begin{problem}
    Дан функтор $\kappa = (\kappa_1, \kappa_2): \mathbf{Cat}(\Gamma) \to 
    \mathbf{Vec}$. \\ 
    Найти $\kappa_2 : (f: \Gamma_1 \to \Gamma_2) \mapsto 
    (A_f: \kappa_1(\Gamma_1) \to \kappa_1(\Gamma_2))$, если известно, что 
    $\kappa_1 : \Gamma \mapsto V$, где $V$ -- пространство характеров, т.е. 
    $V = \{\chi: \Hom \Gamma \to \mathbb{C}: \chi(\psi \circ \phi) = 
    \chi(\psi) + \chi(\phi)\}$.
\end{problem}

Таким образом задача сводится к нахождению линейного оператора $A_f$ на 
коммутативной диаграмме

\[\begin{CD}
    \Gamma_1 @>\kappa>> V_1 \\
    @VVfV          @VVA_fV\\
    \Gamma_2 @>\kappa>> V_2
\end{CD}\]

\begin{proof}[Решение] Для начала отметим три утверждения: если в группоиде 
    $\Gamma$ известны
    \begin{description}
        \item[\mdseries{(1)}] $f: a \to b$, $\hom(a,a)$, то посредством 
        изоморфизма $\psi : \hom(a,a) \to~\hom(b,b)$, а именно $\psi: h \mapsto 
        fhf^{-1}$ однозначно определено $\hom(b,b)$;
        \item[\mdseries{(2)}] $f: a \to b$, $\hom(a,a)$, то однозначно 
        определено $\hom(a,b)$, так как для любого \\$g \in \hom(a,b)$ существует 
        $h \in \hom(a,a)$, такое что $fh = g$, а именно \\$g = 
        f\underbrace{f^{-1}g} = fh$;
        \item[\mdseries{(3)}] $f: a \to b$, $g: a \to c$, то автоматически 
        можно задать $h: b \to c$, а именно $h = gf^{-1}$.
    \end{description}

    Таким образом, если в связном группоиде $\Gamma$ известны группа автоморфизмов 
    $\hom(a,a)$ некоторой вершины $a$ и по одной стрелке $f: a \to b,\: 
    g: a \to c\comdots$ из $a$ в каждую из остальных вершин $b,c\comdots$ то 
    посредством утверджений (1)--(3) однозначно восстанавливается весь группоид 
    $\Gamma$.

    Рассмотрим теперь некоторый характер $\chi: \Hom \Gamma \to \mathbb{C}$.
    Благодаря свойству $\chi(\psi \circ \phi) = \chi(\psi) + \chi(\phi)$ все 
    вышесказанное в определенном смысле переносится и на характер $\chi$. Так 
    если $\chi$ задано на 
    \begin{description}
        \item[\mdseries{(1')}] $f: a \to b$, $\hom(a,a)$, то изоморфизм $\psi$ 
        ``один в один'' переносит харакатер на $\hom (b,b)$: если 
        $\chi(h) = \alpha$, то $\chi(\psi(h)) = \chi(fhf^{-1}) = \chi(f) + 
        \chi(h) - \chi(f) = \chi(h)$, и характер однозначно определен на 
        $\hom(b,b)$.
        \item[\mdseries{(2')}] $f: a \to b$, $\hom(a,a)$, то харктер однозначно 
        продолжается на $\hom(a,b)$, так как для любого $g \in \hom(a,b)$ существует 
        $h \in \hom(a,a)$, такое что $fh = g$, и следовательно $\chi(g) = 
        \chi(f) + \chi(h)$.
        \item[\mdseries{(3')}] $f: a \to b$, $g: a \to c$, то автоматически 
        можно задать характер на некотором $h: b \to c$, а именно $h = gf^{-1}$, 
        и $\chi(h) = \chi(g) - \chi(f)$.
    \end{description}

    Вновь имеем: если в связном группоиде $\Gamma$ определить характер на 
    группе автоморфизмов $\hom(a,a)$ некоторой вершины $a$ и на стрелках $f: a \to b,\: 
    g: a \to c\comdots$ из $a$ (по одной в каждую из остальных вершин $b,c\comdots$), то 
    характер однозначно продолжается на все $\Hom \Gamma$. Так, характер 
    определяется своим  действием на группе автоморфизмов произвольной
    вершины $a$\footnote{или на \emph{фундаментальной группе}, что суть 
    одно и то же,} и вектором значений $s \in \mathbb{C}^{n-1}$ на стрелках из 
    $a$ (здесь $n = |Obj(\Gamma)|$).
    
    \bigskip

    Остановимся на задании характера на некоторой группе $G$, в дальнейшем в 
    качестве $G$ будет рассматриваться фундаментальная группа группоида.

    Как известно\footnote{см. \cite{Vinberg} гл.10 \S 2} разрешимая 
    группа $G$ раскладывается в прямую сумму абелевых групп
    \begin{equation}\label{decomp}
        G \simeq G/G' \oplus \ldots \oplus G^{(n-1)}/G^{(n)},
    \end{equation}
    где $G^{(k+1)} = (G^{(k)})'$ --- коммутант группы $G^{(k)}$.

    Для конечно порожденной абелевой группы $A$
    справедливо разложение\footnote{см.\cite{Vinberg} гл.9 \S 1}
    \[A \simeq \underbrace{\mathbb{Z} \oplus \ldots \oplus \mathbb{Z}}_{n} 
    \oplus \Tor A = \mathbb{Z}^{n} \oplus \Tor A,\]
    где $\Tor A \doteqdot \{a \in A: ma = 0\text{ для некоторого }m \in 
    \mathbb{Z}, m \ne 0\}$ --- \emph{подгруппа кручения}. Более того 
    \[\Tor A = \mathbb{Z}_{u_1} \oplus \ldots \oplus \mathbb{Z}_{u_m}\text{---}\]
    --- сумма примарных подгруп. 

    Таким образом, $A$ представляет собой совокупность всех линейных комбинаций 
    \[k_1 e_1 + \ldots + k_n e_n + k_{n+1} t_1 + \ldots + k_{n+m} t_m 
    \quad (k_i \in \mathbb{Z}),\]
    где $(e_1,\ldots ,e_n)$ --- базис свободной подргуппы, $(t_1,\ldots, t_m)$ 
    --- порождающие примарных подгрупп.

    Так, чтобы задать характер $\chi$ на группе $A$ достаточно 
    определить его значение на $(e_1,\ldots ,e_n)$, $(t_1,\ldots, t_m)$. 
    Впрочем, $\chi(t_i)$ заведомо равно нулю, так как
    \begin{gather*}
        (c_i + 1) t_i = t_i,\\
        \chi((c_i+1) t_i) = \chi(t_i),\\
        (c_i+1)\chi(t_i) = \chi(t_i),\\
        c_i \chi(t_i) = 0,\\
        \chi(t_i) = 0,
    \end{gather*}
    где $c_i > 0$ --- порядок примарной подгруппы, порожденной $t_i$. Отсюда, 
    достаточно задать характер лишь на $(e_1,\ldots ,e_n)$ и он будет 
    однозначно определен на всем $A$.

    %Вернемся к группе $G$ допускающей разложение \eqref{decomp}. Обозначим 
    %абелевы группы $G^{(k)} / G^{(k+1)}$ как $A_k$. В случае если все $A_k$ 
    %конечно порождены, из приведенных рассуждений следует, что для задания 
    %характера на группе $G$ достаточно определить его значения на базисах 
    %свободных подгрупп для каждой $A_k$.

    %Под записью $\dim G$ будем подразумевать число равно сумме размерностей 
    %свободных подпространств всех $A_k$ в разложении \eqref{decomp}. Тогда, 
    %короче: чтобы определить характер $\chi$ группы $G$ достаточно задать 
    %вектор $q\in \mathbb{C}^k$, где $k = \dim G$.

    %\bigskip
    %Говоря о характере группоида мы пришли к тому, что достаточно определить 
    %его на фундаментальной группе $G$ и $n-1$ стрелке. Учитывая сказанное выше, 
    %в действительности 
    %Ка устроено $\kappa(\Gamma)$

    %Что есть $A_f$
\end{proof}
\begin{thebibliography}{0}
    \bibitem{MacLane} Маклейн~С.
        \emph{"<Категории для работающего математика">}. Изд-во ФизМатЛит, Москва, 2004.
    \bibitem{Vinberg} Винберг~Э.~Б.
        \emph{"<Курс алгебры">}. Изд-во МЦНМО, Москва, 2014.
\end{thebibliography}

\end{document}