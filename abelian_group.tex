\subsubsection{Абелева группа}
    Итак, пусть некоторая группа $A$ --- абелева. Как задать на ней характер? 
    Нетрудно получить ответ в случае \emph{конечно-порожденных} групп.

    Известно, что для таких групп справедливо разложение
    \footnote{см.\cite{Vinberg} гл.9 \S 1}
    \begin{equation*}\label{A_decomp}
        A \simeq \underbrace{\mathbb{Z} \oplus \ldots \oplus \mathbb{Z}}_{n} 
    \oplus \Tor A = \mathbb{Z}^{n} \oplus \Tor A,
    \end{equation*}
    где $\mathbb{Z}^{n}$ --- \emph{свободная подгруппа},\\
    $\Tor A \doteqdot \{a \in A: ma = 0\text{ для некоторого }m \in 
    \mathbb{Z}, m \ne 0\}$ --- \emph{подгруппа кручения}, причем
    \begin{equation*}\label{TorA_decomp}
        \Tor A \simeq \mathbb{Z}_{p_1} \oplus \ldots \oplus \mathbb{Z}_{p_s},
    \end{equation*}
    где $\mathbb{Z}_{p_i}$ --- циклическая группа порядка $p_i$.
    
    Отсюда
    \begin{equation}\label{G_basis}
        A = 
        \{x_1 e_1 + \ldots + x_n e_n + x_{n+1} f_1 + \ldots + x_{n+s} f_s 
        \:|\: x_i \in \mathbb{Z}\},
    \end{equation}
    где $\{e_i\}_{i=1}^n$ -- базис свободной подгруппы, $\{f_i\}_{i=1}^s$ -- 
    порождающие соответствующих циклических групп. Попутно введем обозначение 
    $\subdim A = n$.

    Пусть теперь задан характер $\chi: A \to \mathbb{C}$, тогда для любого 
    $a \in A$, с учетом \eqref{G_basis} верно
    \begin{multline*}
        \chi(a) = \chi(\alpha_1 e_1 + \ldots + \alpha_n e_n 
        + \alpha_{n+1} f_1 + \ldots + \alpha_{n+s} f_s ) = \\
        = \alpha_1 \chi(e_1) + \ldots + \alpha_n \chi(e_n) 
        + \alpha_{n+1} \chi(f_1) + \ldots + \alpha_{n+s} \chi(f_s),
    \end{multline*}
    но, так как порядок каждого элемента $f_i$ конечен, то $\chi(f_i) = 0$ для 
    всех $i=1 \comdots, s$, и
    \begin{equation}\label{chi_decomp}
        \chi(a) = \alpha_1 \chi(e_1) + \ldots + \alpha_n \chi(e_n).
    \end{equation}
    
    Тем самым доказано
    \begin{statement}
        Для конечно-порожденной группы $A$ пространство характеров 
        $X(A) = \{\chi: A \to \mathbb{C}: 
        \chi(a + b) = \chi(a) + \chi(b)\}$ имеет размерность
        \begin{equation}
            \dim X(A) = \subdim A.
        \end{equation}
    \end{statement}