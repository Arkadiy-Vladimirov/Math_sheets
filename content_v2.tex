\subsection*{Задача}
    Дан функтор $\kappa = (\kappa_1, \kappa_2): \mathbf{Cat}(\Gamma) \to 
    \mathbf{Vec}$.\\
    Найти $\kappa_2 : (f: \Gamma_1 \to \Gamma_2) \mapsto 
    (A_f: \kappa_1(\Gamma_1) \to \kappa_1(\Gamma_2))$, если известно, что 
    $\kappa_1 : \Gamma \mapsto V$, где $V$ -- пространство характеров, т.е. 
    $V = \{\chi: \Hom \Gamma \to \mathbb{C}: \chi(\psi \circ \phi) = 
    \chi(\psi) + \chi(\phi)\}$.

    Таким образом задача сводится к нахождению линейного оператора $A_f$ на 
    коммутативной диаграмме

    \begin{equation}\label{cd_problem}
    \xymatrix{
        \Gamma_1 \ar[dd]^{\textstyle{f}} \ar[rr]^{\textstyle{\kappa}} & & V_1 \ar[dd]^{\textstyle{A_f}} \\
                                            & & \\
        \Gamma_2 \ar[rr]^{\textstyle{\kappa}}           & & V_2
    }
    \end{equation}

\subsection*{Решение}

\subsubsection{Группоид}
    Для начала отметим следующие три утверждения: если в группоиде $\Gamma$ 
    известны
    \begin{description}
        \item[\mdseries{(a)}] $f: a \to b$, $\hom(a,a)$, то посредством 
        изоморфизма $\psi : \hom(a,a) \to~\hom(b,b)$, а именно $\psi: h \mapsto 
        fhf^{-1}$ однозначно определено $\hom(b,b)$;
        \item[\mdseries{(b)}] $f: a \to b$, $\hom(a,a)$, то однозначно 
        определено $\hom(a,b)$, так как для любого \\$g \in \hom(a,b)$ существует 
        $h \in \hom(a,a)$, такое что $fh = g$, а именно \\$g = 
        f\underbrace{f^{-1}g} = fh$;
        \item[\mdseries{(c)}] $f: a \to b$, $g: a \to c$, то можно задать 
        некоторое $h: b \to c$, а именно $h = gf^{-1}$.
    \end{description}

    Таким образом, если в связном группоиде $\Gamma$ известны группа петель 
    $\hom(a,a)$ некоторой вершины $a$ и по одной стрелке $f: a \to b,\: 
    g: a \to c\comdots$ из $a$ в каждую из остальных вершин $b,c\comdots$ то 
    посредством утверджений (a)--(c) однозначно восстанавливается весь группоид 
    $\Gamma$, что иллюстрирует диаграмма \eqref{cd_Grup}
    \begin{equation}\label{cd_Grup}
    \xymatrix{
        b \ar@{-->}[ddrr]^{\textstyle{h=g \circ f^{-1}}}  &  & \\
        & & \\
        a \ar@(l,d)[]_{\textstyle{\hom(a,a)}} \ar[uu]^{\textstyle{f}} \ar[rr]_{\textstyle{g}} & & c
    }
    \end{equation}

    Рассмотрим теперь некоторый характер $\chi: \Hom \Gamma \to \mathbb{C}$.
    Благодаря свойству $\chi(\psi \circ \phi) = \chi(\psi) + \chi(\phi)$ все 
    вышесказанное в определенном смысле переносится и на характер $\chi$. Так, 
    если $\chi$ задано на 
    \begin{description}
        \item[\mdseries{(a')}] $f: a \to b$, $\hom(a,a)$, то изоморфизм $\psi$ 
        ``один в один'' переносит харакатер на $\hom (b,b)$: если 
        $\chi(h) = \alpha$, то $\chi(\psi(h)) = \chi(fhf^{-1}) = \chi(f) + 
        \chi(h) - \chi(f) = \chi(h)$, и характер однозначно определен на 
        $\hom(b,b)$.
        \item[\mdseries{(b')}] $f: a \to b$, $\hom(a,a)$, то харктер однозначно 
        продолжается на $\hom(a,b)$, так как для любого $g \in \hom(a,b)$ существует 
        $h \in \hom(a,a)$, такое что $fh = g$, и следовательно $\chi(g) = 
        \chi(f) + \chi(h)$.
        \item[\mdseries{(c')}] $f: a \to b$, $g: a \to c$, то автоматически 
        можно задать характер на некотором $h: b \to c$, а именно $h = gf^{-1}$, 
        и $\chi(h) = \chi(g) - \chi(f)$.
    \end{description}

    Таким образом, если в связном группоиде $\Gamma$ определить характер на 
    группе петель $\hom(a,a)$ некоторой вершины $a$ и на стрелках $f: a \to b,\: 
    g: a \to c\comdots$ из $a$ (по одной в каждую из остальных вершин $b,c\comdots$), то 
    характер однозначно продолжается на все $\Hom \Gamma$. То есть, характер 
    определяется своим  действием на группе петель произвольной
    вершины $a$\footnote{или на \emph{фундаментальной группе}, что суть 
    одно и то же,} и вектором значений $s \in \mathbb{C}^{n-1}$ на стрелках из 
    $a$ (здесь $n = |\Obj(\Gamma)|$).

    \begin{equation}\tag{\ref{cd_Grup}'}
        \xymatrix{
            b \ar@{-->}[ddrr]^{\textstyle{\chi(h)=\chi(g)- \chi(f)}}  &  & \\
            & & \\
            a \ar@(l,d)[]_{\textstyle{\chi|_{\hom(a,a)}}} \ar[uu]^{\textstyle{\chi(f)}} \ar[rr]_{\textstyle{\chi(g)}} & & c
        }
    \end{equation}

    \bigskip

    Логично задаться вопросом: как конкретно определяется характер на 
    фунадментальной группе? Для его решения попробуем задать характер на группе 
    вообще.

\subsubsection{Группа}
    Рассмотрим некоторую группу $G$, его фактор-группу $G/G'$ по коммутанту 
    $G'$ и следующую диаграмму

    \begin{equation}\label{cd_ab}
        \xymatrix{
            G \ar[rr]^{\textstyle{\tau}} \ar[rrdd]_{\textstyle{\chi}} & & G/G'\ar[dd]^{\textstyle{\chi_{ab}}} \\
            & & \\
            & & \mathbb{C}
        }
    \end{equation}
    Здесь $\tau: g \mapsto gG'$ --- канонический гомоморфизм; $\chi$, 
    $\chi_{ab}$ --- характеры групп $G$ и $G/G'$ соответственно.

    Оказывается, что 
    \begin{statement} для любого $\chi : G \to \mathbb{C}$ существует и при том 
        единственный характер $\chi_{ab} : G/G' \to \mathbb{C}$ такой, что диаграмма 
        \eqref{cd_ab} коммутативна, т.е. 
        \[\chi = \chi_{ab} \circ \tau.\]
    \end{statement}

    \begin{proof} Действительно, потребуем для любого $g \in G$
    \[\chi(g) = \chi_{ab} \circ \tau (g),\]
    тогда
    \[\chi(g) = \chi_{ab} (gG'),\]
    и $\chi_{ab}$ задан на $G/G'$ однозначно.

    Более того $\chi_{ab}$ задан корректно, т.к. для $\forall f \in gG'$ 
    $\exists h \in G': f = gh$, но по определению коммутанта существуют такие 
    $a$ и $b$, что $h = aba^{-1}b^{-1}$, откуда $f = gaba^{-1}b^{-1}$, и 
    \[\chi(f) = \chi(gaba^{-1}b^{-1}) 
    = \chi(g) + \chi(a) + \chi(b) - \chi(a) - \chi(b) = \chi(g),\]
    то есть,
    \begin{equation}\label{eq_chi_factor}
        \chi(f) = \chi(g),\text{ для любых $f$ и $g$ из одного смежного по $G'$ класса.}
    \end{equation}
    
    Очевидно, что $\chi_{ab}$ --- характер:
    \[\chi_{ab}(gf G') = \chi(gf) = \chi(g) + chi(f) = \chi_{ab}(gG') + \chi_{ab}(fG').\]
    \end{proof}
    
    \begin{remark} Попутно доказано важное для понимания происходящего 
        утверждение \eqref{eq_chi_factor}, показывающее, что факторизация 
        группы по коммутанту $G'$ разбивает ее также и на "<области постоянства"> 
        характера (рис. \ref{img_chi_factor}). Становится яcно, что вместо 
        рассмотрения характера $\chi$ на всей группе, достаточно пронаблюдать 
        лишь за его "<действием с точностью до $G'$">, т.е. за определяемым им 
        на $G/G'$ характере $\chi_{ab}$.
    \end{remark}
    
    \begin{figure}[th]
        \centering
        \includegraphics[width=\textwidth]{pictures/chips}
        \caption{}
        \label{img_chi_factor}
    \end{figure}

    \newpage
    Обратно,
    \begin{statement} характер $\chi_{ab}$ однозначно задает $\chi$, как 
        \[\chi = \chi_{ab}\circ \tau\]
    \end{statement}
    Утверждение представляется очевидным.

    Так, построено взаимооднозначное отображение $t: \chi_{ab} \mapsto 
    \chi_{ab} \circ \tau = \chi$ между характерами группы и ее абелизации 
    (т.е. фактор группы по коммутанту). Покажем, что отображение $t$ является 
    гомоморфизмом (а следовательно и изоморфизмом) линейных пространств.

    Действительно, для любого $g \in G$
        \begin{multline*}
        t(c_1\chi_{ab}^1 + c_2\chi_{ab}^2)(g) 
        = (c_1\chi_{ab}^1 + c_2\chi_{ab}^2) \circ \tau (g) = \\
        = (c_1\chi_{ab}^1 + c_2\chi_{ab}^2) (gG')
        = c_1\chi_{ab}^1 (gG') + c_2\chi_{ab}^2 (gG') = \\
        = c_1\chi_{ab}^1 \circ \tau (g) + c_2\chi_{ab}^2 \circ \tau (g)
        = c_1 t(\chi_{ab}^1)(g) + c_2 t(\chi_{ab}^2)(g).
        \end{multline*}
    Тем самым доказано следующее
    \begin{statement}
        Пространства характеров группы $G$ и ее абелизации $G/G'$ изоморфны. 
        Конкретно, изоморфизм имеет вид:
        \begin{equation}\label{iso_GG'}
            t: G/G' \to G.\quad t: \chi_{ab} \mapsto \chi_{ab} \circ \tau,
        \end{equation}
        где $\tau$ --- канонический гомоморфизм $G \to G/G'$.
    \end{statement}
    
    Последнее утверждение позволяет нам свести задачу изучения характеров
    группы $G$ к рассмотрению характеров на $G/G'$ --- группе, абелевой по 
    определению.

\subsubsection{Абелева группа}
    Итак, пусть некоторая группа $A$ --- абелева. Как задать на ней характер? 
    Нетрудно получить ответ в случае \emph{конечно-порожденных} групп.

    Известно, что для таких групп справедливо разложение
    \footnote{см.\cite{Vinberg} гл.9 \S 1}
    \begin{equation*}\label{A_decomp}
        A \simeq \underbrace{\mathbb{Z} \oplus \ldots \oplus \mathbb{Z}}_{n} 
    \oplus \Tor A = \mathbb{Z}^{n} \oplus \Tor A,
    \end{equation*}
    где $\mathbb{Z}^{n}$ --- \emph{свободная подгруппа},\\
    $\Tor A \doteqdot \{a \in A: ma = 0\text{ для некоторого }m \in 
    \mathbb{Z}, m \ne 0\}$ --- \emph{подгруппа кручения}, причем
    \begin{equation*}\label{TorA_decomp}
        \Tor A \simeq \mathbb{Z}_{p_1} \oplus \ldots \oplus \mathbb{Z}_{p_s},
    \end{equation*}
    где $\mathbb{Z}_{p_i}$ --- циклическая группа порядка $p_i$.
    
    Отсюда
    \begin{equation}\label{G_basis}
        A = 
        \{x_1 e_1 + \ldots + x_n e_n + x_{n+1} f_1 + \ldots + x_{n+s} f_s 
        \:|\: x_i \in \mathbb{Z}\},
    \end{equation}
    где $\{e_i\}_{i=1}^n$ -- базис свободной подгруппы, $\{f_i\}_{i=1}^s$ -- 
    порождающие соответствующих циклических групп. Попутно введем обозначение 
    $\subdim A = n$.

    Пусть теперь задан характер $\chi: A \to \mathbb{C}$, тогда для любого 
    $a \in A$, с учетом \eqref{G_basis} верно
    \begin{multline*}
        \chi(a) = \chi(\alpha_1 e_1 + \ldots + \alpha_n e_n 
        + \alpha_{n+1} f_1 + \ldots + \alpha_{n+s} f_s ) = \\
        = \alpha_1 \chi(e_1) + \ldots + \alpha_n \chi(e_n) 
        + \alpha_{n+1} \chi(f_1) + \ldots + \alpha_{n+s} \chi(f_s),
    \end{multline*}
    но, так как порядок каждого элемента $f_i$ конечен, то $\chi(f_i) = 0$ для 
    всех $i=1 \comdots, s$, и
    \begin{equation}\label{chi_decomp}
        \chi(a) = \alpha_1 \chi(e_1) + \ldots + \alpha_n \chi(e_n).
    \end{equation}
    
    Тем самым доказано
    \begin{statement}
        Для конечно-порожденной группы $A$ пространство характеров 
        $X(A) = \{\chi: A \to \mathbb{C}: 
        \chi(a + b) = \chi(a) + \chi(b)\}$ имеет размерность
        \begin{equation}
            \dim X(A) = \subdim A.
        \end{equation}
    \end{statement}

\qed