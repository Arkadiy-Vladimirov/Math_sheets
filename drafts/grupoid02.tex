\begin{definition}\cite{MacLane}
    \emph{Группоидом} назывется категория, в которой любая стрелка обратима.
\end{definition}

Попытаемся вначале внести ясность в то, как группоид устроен. Под 
группоидом здесь и всюду далее будет подразумеваться связный группоид. 
Введем следующее
\begin{definition}
    \emph{Элементарным группоидом} будем называть группоид для любых двух 
    вершин $a$ и $b$ которого существует одна и притом только одна стрелка 
    $f: a \to b$. см. рис. \ref{cd_eGrup}.
\end{definition}
Тот факт, что такие группоиды существуют доказывается непосредственно 
проверкой аксиом и представляется очевидным.

\begin{figure}[h]
    \centering
    \[\xymatrix{
        b \ar@{-->}[ddrr]^{\textstyle{h=g \circ f^{-1}}} & & \\
        & & \\
        a \ar@(l,d)[]_{\textstyle{\id_a}} \ar[uu]^{\textstyle{f}} \ar[rr]_{\textstyle{g}} & & c
    }\]
    \caption{элементарный группоид}
    \label{cd_eGrup}
\end{figure}

Заметим, что предъявление множества всех вершин и стрелок является 
\emph{избыточным} для задания элементарного группоида. Введем объект 
достаточный (и в некотором смысле минимальный) для определения 
элементарного группоида целиком.

\begin{definition}
    \emph{Пучком стрелок}, исходящих из вершины $a$ назовем совокупность 
    стрелок $f: a \to b$, $g: a \to c,\ldots$, по одной в каждую из 
    остальных вершин $\Obj(\Gamma)/{a}$.
\end{definition}

\begin{statement}
    Элементарный группоид задается пучком стрелок из произвольной вершины. 
    Более точно: пусть $\Gamma$ --- элементарный группоид, $\pi(a)$ --- 
    пучок стрелок из вершины $a \in \Obj(\Gamma)$, тогда минимальный по включению 
    группоид, содержащий $\pi(a)$ совпадает с $\Gamma$.
\end{statement}
\begin{proof}
    Доказательство представялется очевидным. см. рис. \ref{cd_eGrup_bas}
\end{proof}

\begin{figure}[h]
    \centering
    \[\xymatrix{
        b \ar@{.>}@(r,u)[ddrr]^{\textstyle{h=g \circ f^{-1}}} 
          \ar@{.>}@(dl,ul)[dd]_{\textstyle{f^{-1}}} 
        & & \\
        & & \\
        a \ar@{.>}@(l,d)[]_{\textstyle{\id_a = g^{-1}g}} 
          \ar[uu]^{\textstyle{f}} \ar[rr]_{\textstyle{g}} 
        & & 
        c \ar@{.>}@(dl,dr)[ll]^{\textstyle{g^{-1}}} 
          \ar@{.>}@/^/[lluu]_{\textstyle{h^{-1}}}
    }\]
    \caption{элементарный группоид "<натянутый"> на пучок стрелок}
    \label{cd_eGrup_bas}
\end{figure}

Вернемся теперь к группоиду, не обязательно элементарному, и обратим 
внимание на следующий факт 

%\begin{figure}[h]
%    \centering
%    \[\xymatrix{
%        b \ar@{-->}[ddrr]^{\textstyle{h=g \circ f^{-1}}} & & \\
%        & & \\
%        a \ar@(l,d)[]_{\textstyle{\hom(a,a)}} \ar[uu]^{\textstyle{f}} \ar[rr]_{\textstyle{g}} & & c
%    }\]
%    \caption{группоид}
%    \label{cd_Grup}
%\end{figure}

\begin{statement}
    Для любых двух вершин $a$, $b$ $\in \Obj(\Gamma)$ справедливо
    \begin{equation}\label{eq_fhom}
        \hom(a,b) = f \cdot \hom(a,a),
    \end{equation}
    где $f \cdot A \doteqdot \{fh \:|\: \forall h \in A\}$, и 
    $f$ --- произвольная стрелка из $a$ в $b$.
\end{statement}
\begin{proof}
    Вложение правого множества в левое справедливо в силу аксиом 
    композиции в категории.

    Обратное вложение имеет место, так как для любого $g \in \hom(a,b)$ 
    найдется $h \in \hom(a,a)$ такое, что $g = fh$, а именно $h = f^{-1}h$.
\end{proof}

Также общеизвестно слудующее
\begin{statement}
    Все группы петель группоида изоморфны. Конкретно: для любых $a$, $b$ 
    $\in \Obj(\Gamma)$
    \begin{equation}
        \hom(b,b) = h\hom(a,a)h^{-1},
    \end{equation}
    где $h$ --- произвольная стрелка из $a$ в $b$.
\end{statement}

\begin{corollary} 
\end{corollary}

\newpage
\begin{definition}
    назовем \emph{пучком стрелок} исходящим из вершины $a$ совокупность стрелок 
    $f: a \to b$, $g: a \to c,\ldots$, по одной в каждую из остальных вершин 
    $\Obj(\Gamma)/{a}$.
\end{definition}

\begin{definition}
    Назовем \emph{остовом} группоида $\Gamma$ с \emph{основанием} $a$ 
    совокупность группы петель основания $\hom(a,a)$ и пучка исходящих из 
    него стрелок (диаграмма \eqref{cd_ostov}).
\end{definition}

\begin{equation}\label{cd_ostov}
    \xymatrix{
        \bullet  &  & \dots \\
        & & \\
        \bullet \ar@(l,d)[]_{\textstyle{\hom(a,a)}} \ar[uurr] \ar[uu]^{\textstyle{f}} \ar[rr]_{\textstyle{g}} & & \bullet
    }
\end{equation}

Ясно, что
\begin{statement}
    группоид однозначно определяется своим остовом.
\end{statement}

\begin{proof}
    Действительно, пусть дан остов с основанием в вершине $a$, тогда 
    множество объектов группоида определено и состовляет
    \[\Obj(\Gamma) = a \cup \{b = \codom f: \text{по всем $f$ из пучка вершины $a$}\}.\]
    Чтобы показать как остов определяет $\Hom(\Gamma)$ отметим следующие 
    утверждения, справедливые в любом связном группоиде:
    \begin{description}
        \item[\mdseries{(a)}] для любых вершин  $a$  и $b$ 
        \begin{equation}
            \hom(a,b) = f \cdot \hom(a,a) = \{f\cdot h \:|\: \forall h \in \hom(a,a)\},
        \end{equation}
        где $f$ --- некоторая стрелка из $a$ из $b$. Действительно, 
        вложение правого множества в левое очевидно, ввиду аксиом 
        композиции категории. Обратное вложение справедливо, т.к. для 
        любого $g \in \hom(a,b)$ существует $h \in \hom(a,a)$, такое что 
        $fh = g$, а именно $h = f^{-1}g$.
        \item[\mdseries{(b)}]
        \item[\mdseries{(c)}]
    \end{description}

\end{proof}

\bigskip

    Логично задаться вопросом: как конкретно определяется характер на 
    фунадментальной группе? Для его решения попробуем задать характер на группе 
    вообще.
