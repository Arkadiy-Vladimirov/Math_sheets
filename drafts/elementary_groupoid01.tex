\subsubsection{Элементарный группоид}

\begin{definition}
    \emph{Элементарным группоидом} будем называть группоид для любых двух 
    вершин $a$ и $b$ которого существует одна и притом только одна стрелка 
    $f: a \to b$.
\end{definition}

\begin{figure}[h]
    \centering
    \[\xymatrix{
        b \ar@{-->}[ddrr]^{\textstyle{h=g \circ f^{-1}}}                                    & &     \\
                                                                                            & &     \\
        a \ar@(l,d)[]_{\textstyle{\id_a}} \ar[uu]^{\textstyle{f}} \ar[rr]_{\textstyle{g}}   & & c
    }\]
    \caption{элементарный группоид}
    \label{cd_el_func}
\end{figure}

Такие группоиды представляют для нас интерес ввиду простоты своей структуры. 
Более того, такие группоиды можно свести всего навсего к вееру стрелок.

\begin{statement}\label{st_veer}
    Элементарный группоид $E$ однозначно задается любым веером своих стрелок.\\
    Более строго: пусть задан дан веер стрелок $V = \{f : a \to b, g : a \to c. \ldots\}$ 
    вершины $a$ некоторого элементарного группоида $E$. Минимальный по 
    включению группоид, содержащий данный веер есть сам $E$.
\end{statement}

\begin{proof}
    Пусть $\Gamma$ --- Минимальный по включению группоид, содержащий $V$, тогда из 
    определения веера (стрелки проведены из $a$ в каждую из прочих вершин $E$) 
    получаем что $\Obj E \subset \Obj \Gamma$.

    Пользуясь аксиомами композции в категории из стрелок в $V$ можно получить 
    стрелку между любыми двумя вершинами в из $\Obj E$, т.е. по определению 
    элементарного группоида все $\Arr E$. Таким образом, $\Arr E \subset 
    \Arr \Gamma$, а поскольу $\Gamma$ --- минимальный, то $\Gamma = E$.
\end{proof}

    Итак, по вееру стрелок мы можем однозначно восстановить элементарный 
    группоид, но вернемся к вопросу задания характера на группоиде. Из 
    утверждения \ref{st_veer} и аксиом характера следует, что как только 
    характер $\chi$ определен на некотором веере он однозначно продолжается на 
    весь порождаемый им элементарный группоид (рис.~\ref{cd_char_veer}).

    \begin{figure}[h]
        \centering
        \[\xymatrix{
            \cdot                                                   & &       &                                                     & b \ar@{-->}[ddrr]^{\textstyle{h = g \circ f^{-1}}}                                & &  \\
                                                                    & &       & \longrightarrow                                     &                                                                                   & &  \\
            \cdot \ar[uu]^{\textstyle{f}} \ar[rr]_{\textstyle{g}}   & & \cdot &                                                     & a \ar@(l,d)[]_{\textstyle{\id_a}} \ar[uu]^{\textstyle{f}} \ar[rr]_{\textstyle{g}} & & c\\
                                                     &\downarrow_\chi &       &                                                     &                                                                    &\downarrow_\chi &  \\
            \cdot                                                   & &       &                                                     & b \ar@{-->}[ddrr]^{\textstyle{\chi_h = \chi_g - \chi_f}}                                & &  \\
                                                                    & &       & \longrightarrow                                     &                                                                                   & &  \\
            \cdot \ar[uu]^{\textstyle{\chi_f}} \ar[rr]_{\textstyle{\chi_g}}   & & \cdot &                                                     & a \ar@(l,d)[]_{{\chi_f - \chi_f = 0}} \ar[uu]^{\textstyle{\chi_f}} \ar[rr]_{\textstyle{\chi_g}} & & c
        }\]
        \caption{продолжение характера}
        \label{cd_char_veer}
    \end{figure}

    То есть имеет место
    \begin{statement}
        Характер $\chi : \Arr E \to \mathbb{C}$ некоторого элементарного 
        группоида $E$ однозначно задан своим сужением на $V$, где $V$ --- некоторый 
        веер $E$.
    \end{statement}
