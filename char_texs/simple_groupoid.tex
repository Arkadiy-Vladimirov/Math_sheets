\subsubsection{Простой группоид}
    Напомним некоторые свойства характера:
    \begin{itemize}
        \item[a.] $\chi(fg) = \chi(f) + \chi(g)$
        \item[b.] $\chi(f^{-1}) = - \chi(f)$
        \item[c.] $\chi(\id) = 0$
    \end{itemize} 
    
    Как было показано (следствие \ref{cor_simple_grp}) все стрелки простого 
    группоида можно однозначно разложить $v = gf^{-1}$ по некоторому вееру $V$,
    а из свойств a.--c.: $\chi(v) = \chi(g) - \chi(f)$.

    Отсюда ясно, что характер простого группоида 
    однозначно определен $n-1$\footnote{значение на тождественной стрелке 
    автоматически задано нулем} числом~--- его значениями на стрелках некоторого 
    веера. Иначе говоря справедливо

    \begin{statement} Для простого группоида $\Gamma$
        \[X(\Gamma) \simeq \mathbb{C}^{n-1},\]
        где $n$~--- число объектов $\Gamma$.
    \end{statement}

    Разобравшись с первой состовляющей характера группоида (характером 
    простого группоида), перейдем ко второй --- характеру группы.