    Через $X(\Gamma)$ будем обозначать векторное пространство характеров, 
    заданных на некотором группоиде $\Gamma$.
\subsubsection{Группоид}
    Нам потребуется следующая очевидная

    \begin{lemma}\label{lm_char_decomp}
        Для любых двух группоидов $\Gamma_1$ и $\Gamma_2$ справедливо
        \[X(\Gamma_1 \times \Gamma_2) \simeq X(\Gamma_1) \oplus X(\Gamma_2).\]
    \end{lemma}
    \begin{proof}
        В самом деле, для любого $\chi : \Gamma_1 
        \times \Gamma_2 \to \mathbb{C}$, существуют единственные
        $\chi_1 : \Gamma_1 \to \mathbb{C}$, $\chi_2 : \Gamma_2 \to \mathbb{C}$ 
        такие, что диаграмма (рис.~\ref{cd_char_sum}) коммутативна.

        \begin{figure}[h]
            \centering
            \[\xymatrix{
                \Gamma_1 \times \Gamma_2 \ar[rr]^{\textstyle \chi_b \times \chi_c} \ar[dd]_{\textstyle \chi_a}  & & \mathbb{C} \times \mathbb{C} \ar[lldd]^{\textstyle +}\\
                                                                                                                & &                                                      \\
                \mathbb{C}                                                                                      & &
            }\]
            \caption{}
            \label{cd_char_sum}
        \end{figure}
    \end{proof}

    Доказанная лемма вместе с утверждением \ref{st_groupoid_decomp} 
    дают важное
    \begin{statement}[о разложении характера группоида]
        \[X(\Gamma) \simeq X(\Gamma/\Phi_\Gamma) \oplus X(\Phi_\Gamma).\]
    \end{statement}

    Которое позволяет нам вместо рассмотрения характера на группоиде целиком,
    отдельно изучить случаи \emph{простого группоида} ($\Gamma/\Phi_\Gamma$) и
    группы ($\Phi_\Gamma$).

    Первый из них достаточно тривиален.
