\subsubsection{Группоид}
    Сперва обсудим структуру группоида\footnote{здесь и далее под 
    группоидами подразумеваются связные группоиды} ``в целом''.

    \begin{definition}\cite{MacLane}
        \emph{Группоидом} назывется категория, в которой любая стрелка обратима.
    \end{definition}

    \begin{figure}[h]
        \centering
        \[\xymatrix{
            b \ar@(u,l)[]_{\hom(b,b)} \ar[ddrr]^{\hom(b,c)}                     & &                             \\
                                                                                & &                             \\
            a \ar@(l,d)[]_{\hom(a,a)} \ar[uu]^{\hom(a,b)} \ar[rr]_{\hom(a,c)}   & & c
        }\]
        \caption{группоид}
        \label{cd_groupoid}
    \end{figure}

    Для этого попытаемся найти в группоиде ``что-то вроде базиса''. В некотором 
    группоиде $\Gamma$ выберем произвольную вершину $a$ и рассмотрим её группу 
    петель $G$ и \emph{веер стрелок} $(f,g,\ldots)$.

    \begin{definition}
        \emph{Веером стрелок} вершины $a$ группоида $\Gamma$ называется 
        множество стрелок исходящих из вершины $a$ по одной в каждую из прочих.
    \end{definition}

    \begin{figure}[h]
        \centering
        \[\xymatrix{
            b                                                                           & &     \\
                                                                                        & &     \\
            a \ar@(l,d)[]_{\textstyle G} \ar[uu]^{\textstyle f} \ar[rr]_{\textstyle g}  & & c
        }\]
        \caption{группа и веер}
        \label{cd_groupnfan}
    \end{figure}

    Возникает вопрос: как соотносятся с выделенным ``базисом'' остальные 
    стрелки группоида? Ответ на него дает следующая простая лемма.

    \begin{lemma} Для любой пары вершин $b$ и $c$
            \[\hom(b,c) = gGf^{-1},\] 
        где $f : a \to b$, $g : a \to c$, $G = \hom(a,a)$.
    \end{lemma}
    \begin{proof}
        Действительно, вложение $gGf^{-1}$ в $\hom(b,c)$ очевидно ввиду аксиом композиции в 
        категории.

        Обратное вложение доказывается непосредственно: для любой 
        $p \in \hom(b,c)$ существует $h \in G$, такое, что 
        $p = ghf^{-1}$, а именно $h = g^{-1}pf$.
    \end{proof}

    Доказанная лемма дает серию удобных следствий.
    \begin{corollary} для любой вершины $b$
        \begin{itemize}
            \item[a.] $\hom(a,b) = fG$,
            \item[b.] $\hom(b,a) = Gf^{-1}$,
            \item[c.] $\hom(b,b) = fGf^{-1}$, \footnote{Это классическое утверждение об изоморфности 
            всех групп петель в группоиде (именно оно позволяет ввести такой объект как фундаментальная группа)}
        \end{itemize}
        где $G = \hom(a,a)$ и $f : a \to b$.
    \end{corollary}

    Полезно также отедельно выделить частный случай
    \begin{definition}
        \emph{Простым группоидом} назовем группоид, фундаментальная группа 
        которого тривиальна.
    \end{definition}
    Для которого верно очевидное ($G = \id_a$)
    \begin{corollary}
        В простом группоиде для любой стрелки $h : b \to c$ 
        (единственной стрелки из $b$ в $c$ по определению!) справедливо 
        \[h = gf^{-1},\]
        где $f: a \to b$, $g: a \to c$.
    \end{corollary}

    Теперь ясно, что наш ``базис'' $\langle G, (f,g,\ldots) \rangle$, действительно задает весь 
    группоид~--- все стрелки выражаются через него. Наглядно это можно 
    изобразить переходом от рис. \ref{cd_groupoid} к рис. \ref{cd_groupoid_bas}. 
    Последний ``очень напоминает'' некую факторизацию, и действительно, если 
    диаграмму \ref{cd_groupoid_bas} рассматривать не как схематическое 
    изображение группоида, в котором под стрелками подразумеваются $\hom$-множества 
    (выраженные в данном случае через ``базис''), но просто стрелки с названиями 
    типа $gG$, $gGf^{-1}$ и т.д., то мы получим диаграмму \emph{факторизации 
    группоида по его фундаментальной группе}.

    \begin{figure}[h]
        \centering
        \[\xymatrix{
            b \ar@(u,l)[]_{\textstyle fGf^{-1}} \ar[ddrr]^{\textstyle gGf^{-1}}             & &                             \\
                                                                                            & &                             \\
            a \ar@(l,d)[]_{\textstyle G} \ar[uu]^{\textstyle fG} \ar[rr]_{\textstyle gG}    & & c 
        }\]
        \caption{фактор-группоид}
        \label{cd_groupoid_bas}
    \end{figure}

    
    \emph{Фактор-группоид} по фундаментальной группе\footnote{Основываясь на 
    классическом определении\cite{MacLane} фактор-категории, где 
    факторизация проводится исходя из заданного бинарного отношения, 
    действительно можно ввести факторизацию группоида по нормальным подгруппам 
    фундаментальной группы (бинарное отношение получается естественным и 
    тривиальным образом), в частности по ней самой. 
    Мы не будем вводить соответствующие определения, поскольку в нашем случае 
    (факторизация по всей фундаментальной группе) результирующий объект есть 
    всего-навсего \emph{простой группоид} с тем же набором объектов.} мы будем обозначать через $\Gamma/\Phi_\Gamma$.

    \bigskip

    Наконец, при виде диаграммы \ref{cd_groupoid_bas} интуитивно напрашивается 
    вывод:

    \begin{statement}\label{st_groupoid_decomposition}
        \[\textstyle{\Phi_\Gamma \times \Gamma / \Phi_\Gamma \simeq \Gamma}.\]
    \end{statement}

    Здесь $\Gamma$~--- группоид, $\Phi_\Gamma$~--- его фундаментальная 
    группа, $\Gamma / \Phi_\Gamma$~--- фактор-группоид по фундаментальной 
    группе, ``$\times$'' и ``$\simeq$''~--- произведение и изоморфизм категорий 
    соответственно.

    \begin{proof}
        В качестве доказательства данного факта явно построим изоморфизм~--- 
        функтор $\iota : \Phi_\Gamma \times \Gamma / \Phi_\Gamma \to \Gamma$
        
    \end{proof}

