\subsubsection{Группоид}
    Начнем сразу с
    \begin{statement}\label{st_groupoid_decomposition}
        \[\textstyle{\Phi_\Gamma \times \Gamma / \Phi_\Gamma \simeq \Gamma}.\]
    \end{statement}
    Здесь $\Gamma$~--- группоид\footnote{здесь и далее под группоидом будет 
    подразумеваться связный группоид}, $\Phi_\Gamma$~--- его фундаментальная 
    группа, $\Gamma / \Phi_\Gamma$~--- фактор-группоид по фундаментальной 
    группе, ``$\times$'' и ``$\simeq$''~--- произведение и изоморфизм категорий 
    соответственно.

    Вкратце разъясним понятие ``\emph{фактор-группоид по фундаментальной группе}''. 
    Классическое определение фактор-категории\cite{MacLane} подразумевает 
    факторизацию по бинарному отношению, однако как и в группах, где в роли 
    подобного отношения выступает принадлежность смежным классам нормальной 
    подгруппы, в группоидах такое отношение естественно порождается нормальными 
    подгруппами фундаментальной группы, в частном случае ей самой. Мы не будем 
    вводить общее определение факторизации группоида по подгруппе, поскольку в 
    нашем случае факторизации по всей фундаментальной группе результирующий 
    объект является всего-навсего группоидом, с тем же набором 
    объектов и стрелками, отождествляемыми с $\hom$-множествами исходного 
    группоида. Иными словами ``минимальный'' группоид с той же 
    ``пространственной'' структурой, что и исходный, имеющий по одной 
    единственной стрелке $\hom(a,b) : a \to b$ для любых $a$, $b$.

    \begin{figure}[h]
        \centering
        \[\xymatrix{
            b \ar[ddrr]^{\hom(b,c)}                                           & &     \\
                                                                                                & &     \\
            a \ar@(l,d)[]_{\hom(a,a)} \ar[uu]^{\hom(a,b)} \ar[rr]_{\hom(a,c)}   & & c
        }\]
        \caption{фактор-группоид}
        \label{cd_factor_groupoid}
    \end{figure}

    Вернемся теперь к утверждению \ref{st_groupoid_decomposition}.
    \begin{proof}
        Для доказательства утверждения явно построим изоморфизм~--- функтор 
        $\iota : \Phi_\Gamma \times \Gamma / \Phi_\Gamma \to \Gamma$. Задание 
        $\iota$ это в действительности не что иное как своего рода выделение 
        ``базиса'' группоида. Действительно, выберем некоторый объект $a$ 
        группоида $\Gamma$, выделим группу его петель $G$, и некоторый веер
        \footnote{\textbf{Определение.} \emph{Веером стрелок} вершины 
        $a$ группоида $\Gamma$ назовем множество состоящее из стрелок исходящих 
        из вершины $a$ по одной в каждую из прочих.} ее стрелок $(f, g, h, \ldots)$.
    \end{proof}

    %\begin{figure}[h]
    %    \centering
    %    \[\xymatrix{
    %        \Phi_\Gamma \times \Gamma / \Phi_\Gamma \ar[rr]^{\textstyle{\tau}} \ar[rrdd]_{\textstyle{\chi}}    & & \Gamma\ar[dd]^{\textstyle{\chi}}    \\
    %                                                                                                                & &                                     \\
    %                                                                                                                & & \mathbb{C}
    %    }\]
    %    \caption{}
    %    \label{cd_groupoidchar}
    %\end{figure}