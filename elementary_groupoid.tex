\subsubsection{Элементарный группоид}
Сперва попытаемся задать характер не на группоидах\footnote{здесь и далее под 
группоидами подразумеваются связные группоиды} вообще:
\begin{definition}\cite{MacLane}
    \emph{Группоидом} назывется категория, в которой любая стрелка обратима.
\end{definition}

Но на группоидах простого вида.
\begin{definition}
    \emph{Элементарным группоидом} будем называть группоид для любых двух 
    вершин $a$ и $b$ которого существует одна и притом только одна стрелка 
    $f: a \to b$.
\end{definition}
Тот факт, что такие группоиды существуют доказывается непосредственно 
проверкой аксиом и представляется очевидным.