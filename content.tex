\begin{problem}
    Дан функтор $\kappa = (\kappa_1, \kappa_2): \mathbf{Cat}(\Gamma) \to 
    \mathbf{Vec}$. \\ 
    Найти $\kappa_2 : (f: \Gamma_1 \to \Gamma_2) \mapsto 
    (A_f: \kappa_1(\Gamma_1) \to \kappa_1(\Gamma_2))$, если известно, что 
    $\kappa_1 : \Gamma \mapsto V$, где $V$ -- пространство характеров, т.е. 
    $V = \{\chi: \Hom \Gamma \to \mathbb{C}: \chi(\psi \circ \phi) = 
    \chi(\psi) + \chi(\phi)\}$.
\end{problem}

Таким образом задача сводится к нахождению линейного оператора $A_f$ на 
коммутативной диаграмме

\[\begin{CD}
    \Gamma_1 @>\kappa>> V_1 \\
    @VVfV          @VVA_fV\\
    \Gamma_2 @>\kappa>> V_2
\end{CD}\]

\begin{proof}[Решение] Для начала отметим три утверждения: если в группоиде 
    $\Gamma$ известны
    \begin{description}
        \item[\mdseries{(1)}] $f: a \to b$, $\hom(a,a)$, то посредством 
        изоморфизма $\psi : \hom(a,a) \to~\hom(b,b)$, а именно $\psi: h \mapsto 
        fhf^{-1}$ однозначно определено $\hom(b,b)$;
        \item[\mdseries{(2)}] $f: a \to b$, $\hom(a,a)$, то однозначно 
        определено $\hom(a,b)$, так как для любого \\$g \in \hom(a,b)$ существует 
        $h \in \hom(a,a)$, такое что $fh = g$, а именно \\$g = 
        f\underbrace{f^{-1}g} = fh$;
        \item[\mdseries{(3)}] $f: a \to b$, $g: a \to c$, то автоматически 
        можно задать $h: b \to c$, а именно $h = gf^{-1}$.
    \end{description}

    Таким образом, если в связном группоиде $\Gamma$ известны группа автоморфизмов 
    $\hom(a,a)$ некоторой вершины $a$ и по одной стрелке $f: a \to b,\: 
    g: a \to c\comdots$ из $a$ в каждую из остальных вершин $b,c\comdots$ то 
    посредством утверджений (1)--(3) однозначно восстанавливается весь группоид 
    $\Gamma$.

    Рассмотрим теперь некоторый характер $\chi: \Hom \Gamma \to \mathbb{C}$.
    Благодаря свойству $\chi(\psi \circ \phi) = \chi(\psi) + \chi(\phi)$ все 
    вышесказанное в определенном смысле переносится и на характер $\chi$. Так 
    если $\chi$ задано на 
    \begin{description}
        \item[\mdseries{(1')}] $f: a \to b$, $\hom(a,a)$, то изоморфизм $\psi$ 
        ``один в один'' переносит харакатер на $\hom (b,b)$: если 
        $\chi(h) = \alpha$, то $\chi(\psi(h)) = \chi(fhf^{-1}) = \chi(f) + 
        \chi(h) - \chi(f) = \chi(h)$, и характер однозначно определен на 
        $\hom(b,b)$.
        \item[\mdseries{(2')}] $f: a \to b$, $\hom(a,a)$, то харктер однозначно 
        продолжается на $\hom(a,b)$, так как для любого $g \in \hom(a,b)$ существует 
        $h \in \hom(a,a)$, такое что $fh = g$, и следовательно $\chi(g) = 
        \chi(f) + \chi(h)$.
        \item[\mdseries{(3')}] $f: a \to b$, $g: a \to c$, то автоматически 
        можно задать характер на некотором $h: b \to c$, а именно $h = gf^{-1}$, 
        и $\chi(h) = \chi(g) - \chi(f)$.
    \end{description}

    Вновь имеем: если в связном группоиде $\Gamma$ определить характер на 
    группе автоморфизмов $\hom(a,a)$ некоторой вершины $a$ и на стрелках $f: a \to b,\: 
    g: a \to c\comdots$ из $a$ (по одной в каждую из остальных вершин $b,c\comdots$), то 
    характер однозначно продолжается на все $\Hom \Gamma$. Так, характер 
    определяется своим  действием на группе автоморфизмов произвольной
    вершины $a$\footnote{или на \emph{фундаментальной группе}, что суть 
    одно и то же,} и вектором значений $s \in \mathbb{C}^{n-1}$ на стрелках из 
    $a$ (здесь $n = |\Obj(\Gamma)|$).
    
    \bigskip

    Проясним как задается характер на фундментальной группе. Для этого 
    остановимся на определении характера на некоторой группе $G$.

    Как известно\footnote{см. \cite{Vinberg} гл.10 \S 2} разрешимая 
    группа $G$ раскладывается в прямую сумму
    \begin{equation}\label{G_decomp}
        G \simeq G/G' \oplus \ldots \oplus G^{(n-1)}/G^{(n)},
    \end{equation}
    где $G^{(k+1)} = (G^{(k)})'$ --- коммутант группы $G^{(k)}$. 

    Для конечно порожденной абелевой группы $A$
    справедливо разложение\footnote{см.\cite{Vinberg} гл.9 \S 1}
    \begin{equation}\label{A_decomp}
        A \simeq \underbrace{\mathbb{Z} \oplus \ldots \oplus \mathbb{Z}}_{n} 
    \oplus \Tor A = \mathbb{Z}^{n} \oplus \Tor A,
    \end{equation}
    где $\Tor A \doteqdot \{a \in A: ma = 0\text{ для некоторого }m \in 
    \mathbb{Z}, m \ne 0\}$ --- \emph{подгруппа кручения}, причем
    \begin{equation}\label{TorA_decomp}
        \Tor A \simeq \mathbb{Z}_{p_1} \oplus \ldots \oplus \mathbb{Z}_{p_s},
    \end{equation}
    где $\mathbb{Z}_{p}$ --- циклическая группа порядка $p$.

    \begin{definition}
        Назовем группу $G$ \emph{конечно разрешимой}, если она разрешима и 
        каждое слагаемое разложения \eqref{G_decomp} суть конечно порожденная 
        абелева группа.
    \end{definition}
    
    Таким образом, из соотношений \eqref{G_decomp} -- \eqref{TorA_decomp} 
    следует, что если $G$ конечно разрешимая группа, то
    \begin{gather*}
        G \simeq A_0 \oplus \ldots \oplus A_{n-1} \simeq \\ 
        \simeq \mathbb{Z}^m \oplus \Tor A_0 \oplus \ldots \oplus \Tor A_{n-1} \simeq \\
        \simeq \mathbb{Z}^m \oplus \mathbb{Z}_{p_1} \oplus \ldots \oplus \mathbb{Z}_{p_k},
    \end{gather*}
    т.е. разложима в сумму конечных и бесконечных циклических групп, а значит
    \begin{equation}\label{G_basis}
        G = 
        \{x_1 e_1 + \ldots + x_m e_m + x_{m+1} f_1 + \ldots + x_{m+k} f_k 
        \:|\: x_i \in \mathbb{Z}\},
    \end{equation}
    где $\{e_i\}_1^m$ -- базис свободной группы $\mathbb{Z}^m$, $\{f_i\}_1^k$ -- 
    порождающие соответствующих циклических групп $\mathbb{Z}_{p_i}$. Попутно 
    введем обозначение $\subdim{G} = m$.

    Пусть теперь задан характер $\chi: G \to \mathbb{C}$, тогда для любого 
    $g \in G$, с учетом \eqref{G_basis} верно
    \begin{multline*}
        \chi(g) = \chi(x_1 e_1 + \ldots + x_m e_m 
        + x_{m+1} f_1 + \ldots + x_{m+k} f_k ) = \\
        = x_1 \chi(e_1) + \ldots + x_m \chi(e_m) 
        + x_{m+1} \chi(f_1) + \ldots + x_{m+k} \chi(f_k),
    \end{multline*}
    но, так как порядок каждого элемента $f_i$ конечен, то $\chi(f_i) = 0$ для 
    всех $i=1 \comdots, k$, и
    \begin{equation}\label{chi_decomp}
        \chi(g) = x_1 \chi(e_1) + \ldots + x_m \chi(e_m).
    \end{equation}
    Так, характер конечно разрешимой группы $G$ определяется $m = \subdim{G}$ числами~--- 
    значениями характера на базисе свободной подгруппы.

    \bigskip

    Вернемся к характеру группоида $\Gamma$. Как было показано ранее он 
    определен $(\Obj \Gamma) - 1$ числами и своим действием на фундаментальной 
    группе группоида $\Fund \Gamma$. Теперь, с учетом \eqref{chi_decomp}, ясно:
    \emph{характер $\chi :\Hom \Gamma \to \mathbb{C}$ связного группоида 
    $\Gamma$, фундаментальная группа которого конечно разрешима, определяется 
    своими значениями на $n = |\Obj \Gamma| - 1$ стрелках исходящих из некоторой 
    вершины в все прочие и $m = \subdim{\Fund \Gamma}$ значениями на базисе 
    свободной подгруппы фундаментальной подгруппы.} То есть, вектором 
    $h(\chi) \in \mathbb{C}^{n+m}$.

    Теперь, после того как мы можем взаимооднозначно сопоставить любому 
    характеру вектор пространства соответствующей размерности, приходим к 
    очевидному выводу: 
    \begin{equation}
        V \doteqdot \{\chi: \Hom \Gamma \to \mathbb{C}\} \simeq 
        \mathbb{C}^{\subdim{\Fund \Gamma} + |\Obj \Gamma| - 1},
    \end{equation}
    где $\Gamma$ --- группоид, такой что $|\Obj \Gamma| < \infty$ и 
    $\Fund \Gamma$ конечно разрешима.



    

    %Каждая из групп $A_k = G^{(k)}/G^{(k+1)}$ абелева, если к тому же все 
    %они кончено порождены, то для произвольно выбранной $A = A_k$ имеет место 
    %разложение\footnote{см.\cite{Vinberg} гл.9 \S 1}
    %\[A \simeq \underbrace{\mathbb{Z} \oplus \ldo3ts \oplus \mathbb{Z}}_{p} 
    %\oplus \Tor A = \mathbb{Z}^{p} \oplus \Tor A,\]
    %и 
    %\begin{equation}\label{bas_A}
    %    A = \langle e_1,\ldots,e_p, f_1,\ldots, f_q\rangle,
    %\end{equation}
    %где $e$ --- базис 
    %свободной подгруппы $\mathbb{Z}^p$, a $f$ --- порождающие примарных 
    %подгрупп, прямая сумма которых есть $\Tor A$.

    %Поскольку $G \simeq A_0 \oplus \ldots \oplus A_{n-1}$, и \eqref{bas_A} 
    %верно для всех $A_k$, справедливо
    %\[G = \langle e_1,\ldots,e_p, f_1,\ldots, f_q\rangle,\]
    %где $e$ --- объединение базисов всех свободных подгрупп $A_k$, $f$ --- 
    %порождающих соответствующих примарных групп, p

    %Для конечно порожденной абелевой группы $A$
    %справедливо разложение\footnote{см.\cite{Vinberg} гл.9 \S 1}
    %\[A \simeq \underbrace{\mathbb{Z} \oplus \ldots \oplus \mathbb{Z}}_{n} 
    %\oplus \Tor A = \mathbb{Z}^{n} \oplus \Tor A,\]
    %где $\Tor A \doteqdot \{a \in A: ma = 0\text{ для некоторого }m \in 
    %\mathbb{Z}, m \ne 0\}$ --- \emph{подгруппа кручения}. Более того 
    %\[\Tor A = \mathbb{Z}_{u_1} \oplus \ldots \oplus \mathbb{Z}_{u_m}\text{---}\]
    %--- сумма примарных подгруп. 

    %Таким образом, $A$ представляет собой совокупность всех линейных комбинаций 
    %\[k_1 e_1 + \ldots + k_n e_n + k_{n+1} t_1 + \ldots + k_{n+m} t_m 
    %\quad (k_i \in \mathbb{Z}),\]
    %где $(e_1,\ldots ,e_n)$ --- базис свободной подргуппы, $(t_1,\ldots, t_m)$ 
    %--- порождающие примарных подгрупп.

    %Так, чтобы задать характер $\chi$ на группе $A$ достаточно 
    %определить его значение на $(e_1,\ldots ,e_n)$, $(t_1,\ldots, t_m)$. 
    %Впрочем, $\chi(t_i)$ заведомо равно нулю, так как
    %\begin{gather*}
    %    (c_i + 1) t_i = t_i,\\
    %    \chi((c_i+1) t_i) = \chi(t_i),\\
    %    (c_i+1)\chi(t_i) = \chi(t_i),\\
    %    c_i \chi(t_i) = 0,\\
    %    \chi(t_i) = 0,
    %\end{gather*}
    %где $c_i > 0$ --- порядок примарной подгруппы, порожденной $t_i$. Отсюда, 
    %достаточно задать характер лишь на $(e_1,\ldots ,e_n)$ и он будет 
    %однозначно определен на всем $A$.

    %Вернемся к группе $G$ допускающей разложение \eqref{decomp}. Обозначим 
    %абелевы группы $G^{(k)} / G^{(k+1)}$ как $A_k$. В случае если все $A_k$ 
    %конечно порождены, из приведенных рассуждений следует, что для задания 
    %характера на группе $G$ достаточно определить его значения на базисах 
    %свободных подгрупп для каждой $A_k$.

    %Под записью $\dim G$ будем подразумевать число равно сумме размерностей 
    %свободных подпространств всех $A_k$ в разложении \eqref{decomp}. Тогда, 
    %короче: чтобы определить характер $\chi$ группы $G$ достаточно задать 
    %вектор $q\in \mathbb{C}^k$, где $k = \dim G$.

    %\bigskip
    %Говоря о характере группоида мы пришли к тому, что достаточно определить 
    %его на фундаментальной группе $G$ и $n-1$ стрелке. Учитывая сказанное выше, 
    %в действительности 
    %Ка устроено $\kappa(\Gamma)$

    %Что есть $A_f$
\end{proof}