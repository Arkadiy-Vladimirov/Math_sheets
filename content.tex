\begin{problem}
    Дан функтор $\kappa = (\kappa_1, \kappa_2): \mathbf{Cat}(\Gamma) \to 
    \mathbf{Vec}$. \\ 
    Найти $\kappa_2 : (f: \Gamma_1 \to \Gamma_2) \mapsto 
    (A_f: \kappa_1(\Gamma_1) \to \kappa_1(\Gamma_2))$, если известно, что 
    $\kappa_1 : \Gamma \mapsto V$, где $V$ -- пространство характеров, т.е. 
    $V = \{\chi: \Hom \Gamma \to \mathbb{C}: \chi(\psi \circ \phi) = 
    \chi(\psi) + \chi(\phi)\}$.
\end{problem}

Таким образом задача сводится к нахождению линейного оператора $A_f$ на 
коммутативной диаграмме

\[\begin{CD}
    \Gamma_1 @>\kappa>> V_1 \\
    @VVfV          @VVA_fV\\
    \Gamma_2 @>\kappa>> V_2
\end{CD}\]

\begin{proof}[Решение] Для начала отметим три утверждения: если в группоиде 
    $\Gamma$ известны
    \begin{description}
        \item[\mdseries{(1)}] $f: a \to b$, $\hom(a,a)$, то посредством 
        изоморфизма $\psi : \hom(a,a) \to~\hom(b,b)$, а именно $\psi: h \mapsto 
        fhf^{-1}$ однозначно определено $\hom(b,b)$;
        \item[\mdseries{(2)}] $f: a \to b$, $\hom(a,a)$, то однозначно 
        определено $\hom(a,b)$, так как для любого \\$g \in \hom(a,b)$ существует 
        $h \in \hom(a,a)$, такое что $fh = g$, а именно \\$g = 
        f\underbrace{f^{-1}g} = fh$;
        \item[\mdseries{(3)}] $f: a \to b$, $g: a \to c$, то автоматически 
        можно задать $h: b \to c$, а именно $h = gf^{-1}$.
    \end{description}

    Таким образом, если в связном группоиде $\Gamma$ известны группа автоморфизмов 
    $\hom(a,a)$ некоторой вершины $a$ и по одной стрелке $f: a \to b,\: 
    g: a \to c\comdots$ из $a$ в каждую из остальных вершин $b,c\comdots$ то 
    посредством утверджений (1)--(3) однозначно восстанавливается весь группоид 
    $\Gamma$.

    Рассмотрим теперь некоторый характер $\chi: \Hom \Gamma \to \mathbb{C}$.
    Благодаря свойству $\chi(\psi \circ \phi) = \chi(\psi) + \chi(\phi)$ все 
    вышесказанное в определенном смысле переносится и на характер $\chi$. Так 
    если $\chi$ задано на 
    \begin{description}
        \item[\mdseries{(1')}] $f: a \to b$, $\hom(a,a)$, то изоморфизм $\psi$ 
        ``один в один'' переносит харакатер на $\hom (b,b)$: если 
        $\chi(h) = \alpha$, то $\chi(\psi(h)) = \chi(fhf^{-1}) = \chi(f) + 
        \chi(h) - \chi(f) = \chi(h)$, и характер однозначно определен на 
        $\hom(b,b)$.
        \item[\mdseries{(2')}] $f: a \to b$, $\hom(a,a)$, то харктер однозначно 
        продолжается на $\hom(a,b)$, так как для любого $g \in \hom(a,b)$ существует 
        $h \in \hom(a,a)$, такое что $fh = g$, и следовательно $\chi(g) = 
        \chi(f) + \chi(h)$.
        \item[\mdseries{(3')}] $f: a \to b$, $g: a \to c$, то автоматически 
        можно задать характер на некотором $h: b \to c$, а именно $h = gf^{-1}$, 
        и $\chi(h) = \chi(g) - \chi(f)$.
    \end{description}

    Вновь имеем: если в связном группоиде $\Gamma$ определить характер на 
    группе автоморфизмов $\hom(a,a)$ некоторой вершины $a$ и на стрелках $f: a \to b,\: 
    g: a \to c\comdots$ из $a$ (по одной в каждую из остальных вершин $b,c\comdots$), то 
    характер однозначно продолжается на все $\Hom \Gamma$. Так, характер 
    определяется своим  действием на группе автоморфизмов произвольной
    вершины $a$\footnote{или на \emph{фундаментальной группе}, что суть 
    одно и то же,} и вектором значений $s \in \mathbb{C}^{n-1}$ на стрелках из 
    $a$ (здесь $n = |Obj(\Gamma)|$).

    Остановимся на задании характера на некоторой группе $G$, в дальнейшем в 
    качестве $G$ будет рассматриваться фундаментальная группа группоида.

    Как известно\footnote{см. \cite{Vinberg} гл.10 \S 2} разрешимая 
    группа $G$ раскладывается в прямую сумму абелевых групп 
    \[G = G/G' \oplus \ldots \oplus G^{(n-1)}/G^{(n)},\]
    где $G^{(k+1)} = (G^{(k)})'$ --- коммутант группы $G^{(k)}$.

    Обозначим $G^{(k)} / G^{(k+1)}$ как $A_k$. Так как $A_k$ --- абелева, 
    справедливо ее разложение\footnote{см.\cite{Vinberg} гл.9 \S 1}
    \[A_k = \underbrace{\mathbb{Z} \oplus \ldots \oplus \mathbb{Z}}_{n_k} 
    \oplus \Tor A_k = \mathbb{Z}^{n_k} \oplus \Tor A_k.\]



    %Ка устроено $\kappa(\Gamma)$

    %Что есть $A_f$
\end{proof}